% !TEX TS-program = pdflatex
%
% Created by João Lourenço on 2021-03-23.
% Copyright (c) 2021 .
\documentclass[a4paper,11pt]{article}


%-------------------------------------------------------------
% Configuração do pacote "novafctletterhead"
%-------------------------------------------------------------
% \usepackage{novafctletterhead}  % defaults: times new roman font, black headings
% \usepackage[everypage,opensans,blue]{novafctletterhead}  % With blue section headings
\usepackage[everypage,opensans]{novafctletterhead}
% opensans é uma das novas fontes oficiais para a NOVA FCT

% \fctdepartment{Departamento de\\Ciência dos Materiais}{Department of\\Materials Science}
% \fctdepartment{Departamento de\\Ciências Sociais Aplicadas}{Department of\\Applied Social Sciences}
% \fctdepartment{Departamento de\\Ciências da Terra}{Department of\\Earth Sciences}
% \fctdepartment{Departamento de\\Ciências da Vida}{Department of\\Life Sciences}
% \fctdepartment{Departamento de\\Ciências e Engenharia do Ambiente}{Department of\\Environmental Sciences and Engineering}
% \fctdepartment{Departamento de\\Ciências e Tecnologia da Biomassa}{Department of\\Sciences and Technology of Biomass}
% \fctdepartment{Departamento de\\Conservação e Restauro}{Department of\\Conservation and Restoration}
% \fctdepartment{Departamento de\\Engenharia Civil}{Department of\\Civil Engineering}
% \fctdepartment{Departamento de Engenharia\\Eletrotécnica e de Computadores}{Department of Electrical\\and Computer Engineering}
% \fctdepartment{Departamento de\\Engenharia Mecânica e Industrial}{Department of\\Mechanical and Industrial Engineering}
% \fctdepartment{Departamento de\\Física}{Department of\\Physics}
\fctdepartment{Departamento de\\Informática}{Department of\\Computer Science}
% \fctdepartment{Departamento de\\Matemática}{Department of\\Mathematics}
% \fctdepartment{Departamento de\\Química}{Department of\\Chemistry}
\fctphone{+351 212 948 536}
% \fctext{10740}
% \fctemail{joao.lourenco@fct.unl.pt}
% \fcturl{https://docentes.fct.unl.pt/joao-lourenco}




%-------------------------------------------------------------
% Resto do docuemnto LaTeX
%-------------------------------------------------------------
\usepackage[T1]{fontenc}
\usepackage{calligra}
\usepackage{courier}
\usepackage[portuguese]{babel}
\usepackage{xltabular}
\usepackage[colorlinks]{hyperref}
\providecolor{fctblue}{RGB}{22,93,149}
\hypersetup{allcolors=fctblue}
% \geometry{hmargin=2.0cm,vmargin=2.5cm}

\newcommand*{\thePackage}{\texttt{\novafctletterheadname}}

\fcttitle{O pacote \LaTeX\ \thePackage}
\fctauthor{João M. Lourenço, Professor Associado}
% \fctauthor{John Doe, Professor Auxiliar}      % descomentar para ver múltiplas assinaturas
% \fctauthor{Alice Doe, Professora Catedrático} % descomentar para ver múltiplas assinaturas
\date{\novafctdate\ \ (v.\ \novafctversion)}
\fctauthorsigfont{\LARGE\color{fctblue}\calligra}
\fctpositionsigfont{\scriptsize}

\title{\theTitle}
\author{\theAuthor}

\hypersetup{
  pdftitle   = {\theTitle},
  pdfsubject = {\theDepartment $|$ NOVA FCT},
  pdfauthor  = {\theAuthor}
}


% The document
\begin{document}

\maketitle

% \thispagestyle{empty}% Não imprimir número de página na primeira página

\begin{abstract}
    Este documento é simultaneamente um manual de instruções e um exemplo de como usar o pacote “\thePackage”.  Este pacote permite produzir documentos em \emph{papel letterhead} da FCT-NOVA.
\end{abstract}


\section{Preâmbulo}

A versão mais recente deste pacote está disponível para download em:

\begin{center}
  \url{https://github.com/joaomlourenco/novafctletterhead}
\end{center}

Se não tem a certeza se está a utilizar a versão mais recente, aproveite e \href{https://github.com/joaomlourenco/novafctletterhead/archive/refs/heads/main.zip}{faça download} da última versão!   Já agora, se achar este pacote útil, \href{https://www.paypal.com/donate/?hosted_button_id=8WA8FRVMB78W8}{ofereça um café} ao meu alter-ego \emph{NOVAthesis} e na caixa de comentários diga que é para o pacote \thePackage! ;)


\section{Usar o Pacote \thePackage}

Na verdade é muito simples usar este pacote \thePackage…  basta adicionar no preâmbulo do ficheiro fonte, i.e., depois do \verb!\docuemntclass{…}! e antes do \verb!\begin{document}!, o seguinte comando:

\begin{verbatim}
  \usepackage[opções]{novafctletterhead}
\end{verbatim}

\noindent onde, por omissão, apenas a primeira página será timbrada.

\medskip
As opções válidas são:\vspace{-1.5ex}

\bgroup
  \renewcommand{\arraystretch}{1.5}
  \begin{xltabular}{\textwidth}{lX}
    \texttt{everypage}  & Todas as páginas serão timbradas e não apenas a primeira.\\
    \texttt{opensans} & Utilizar a fonte \emph{OpenSans}, que é uma das fontes oficiais/recomendadas da NOVA FCT.\\
  \end{xltabular}
\egroup

\section{Configurar o Pacote \thePackage}

Depois, ainda no preâmbulo, deverá configurar os seus dados e os do seu departamento

\bgroup
  \renewcommand{\arraystretch}{1.5}
  \begin{xltabular}{\textwidth}{lX}
    \verb+\fctdepartment+ & Este comando recebe dois argumentos, o primeiro com o nome do departamento em Português e o segundo com o nome do departamento em Inglês, a serem apresentados no canto superior direito (convertidos em maiúsculas).  \textbf{\textsl{Se omitido nada será apresentado em cima à direita.}}\\
    \verb+\fctphone+   & Este comando recebe como argumento o telefone do autor, a colocar no rodapé (em baixo à direita). \textbf{\textsl{Se omitido apresentará o telefone geral da FCT-NOVA.}}\\
    \verb+\fctext+   & Este comando recebe como argumento uma extensão telefónica, a colocar no rodapé à direita do telefone. \textbf{\textsl{Se omitido será apresentado apenas o número de telefone (sem extensão).}}\\
    \verb+\fctemail+  & Este comando recebe como argumento um email, a colocar no rodapé por baixo do telefone/extensão. \textbf{\textsl{Se omitido nada será apresentado.}}\\
    \verb+\fcturl+   & Este comando recebe como argumento um url, a colocar no rodapé por baixo do email. \textbf{\textsl{Se omitido nada será apresentado.}}\\
    \verb+fcttitle+   & Título do documento.\\
    \verb+\fctauthor+     & Este comando recebe como argumento o nome do autor e, opcionalmente, o título/posição, a colocar na zona de assinatura.  \textbf{\textsl{Se omitido nada será apresentado.  Se usado múltiplas vezes, criará zonas de assinatura para cada um dos autores.}}\\
    \verb+\fctauthorsigfont+  &  Configurar a fonte para escrever o(s) nome(s) do(s) autor(es).\\
    \verb+\fctpositionsigfont+  &  Configurar a fonte para escrever a(s) posições(s) do(s) autor(es).\\
  \end{xltabular}
\egroup

\section{Configurar e Apresentar a Zona de Assinatura}

Há também um comando para criar uma zona para colocar uma assinatura.  Este comando tem três argumentos, sendo que o primeiro (entre “[…]”) é opcional.

\begin{verbatim}
  \fctsignature[tamanho_da_linha]{posição}{afastamento}
\end{verbatim}

\noindent onde:

\medskip
\bgroup
  \renewcommand{\arraystretch}{1.5}
  \begin{xltabular}{\textwidth}{lX}
    \texttt{tamanho\_da\_linha}  & \emph{Opcional} — Indicação do tamanho da linha (em qualquer unidade válida no \LaTeX, por exemplo, \emph{6cm}, \emph{2in}, \emph{30pt}, etc).  \textbf{\textsl{Se omitido desenha uma linha ligeiramente mais larga que o nome/posição.}}\\
    \texttt{posição} & \emph{Obrigatório} — Indicação da localização da zona de assinatura.  Valores possíveis:\newline
    \begin{tabular}[t]{>{\slshape\bfseries}ll}
      l & zona de assinatura à esquerda;\\
      c & zona de assinatura centrada; e\\
      r & zona de assinatura à direita.\\
    \end{tabular}\\
    \texttt{afastamento}  & \emph{Obrigatório} — Indicação do espaço a deixar entre o final do texto e a zona de assinatura (em qualquer unidade válida no \LaTeX, por exemplo, \emph{3cm}, \emph{2.5in}, \emph{10ex}, etc).\\
  \end{xltabular}
\egroup

A zona de assinatura que termina este documento, localizada em baixo à direita, foi criada com:

\begin{verbatim}
  \fctauthor{João M. Lourenço, Professor Associado}
  \fctauthorsigfont{\LARGE\calligra}
  \fctpositionsigfont{\scriptsize}
  ...
  \fctsignature{r}{1.5cm}
\end{verbatim}

\fctsignature{r}{1.5cm}
% \fctsignature[7cm]{r}{2cm}


\end{document}

